
\documentclass{article}

\usepackage[spanish]{babel} % De manera obligatoria usamos babel
\usepackage[utf8]{inputenc} % Para agregar la codificacion de escritura del archivo.
\usepackage{xcolor}
\usepackage{ulem}
\usepackage{soul}
% Dos paquetes importantes para el ambiente matemático
\usepackage{amsmath}
\usepackage{amssymb}
\usepackage{graphicx} % Para incluir imagenes
\usepackage{vmargin}
\usepackage{txfonts} %Para las integrales cerradas, podemos usar las instrucciones \oint, \oiint, \oiiint (para integrales sencillas, dobles y triples). En el caso de las dos últimas, necesitaremos cargar el paquete {txfonts} o {pxfonts}.



\begin{document}
	$ (a^2 + b^2) = c^2$\\
	
	$ c = \sqrt[]{(a^2 * b^2)}$\\
	
	$ A \oplus \normalsize B = A\overline{B} + \overline{A}B$ \\
	
	El valor de: $R_1$ es: $300[\Omega]$\\
	
	$Z(X) = \frac{X- \mu}{\sigma}$\\
	
	$\sum_{i = 0}^{n} n = \frac{n(n+1)}{2}$\\
	
	$\sin^2 x + \cos^2x = 1$\\
	
	$F(\omega) = \int_{\infty}^{\infty} f(t)e^{-j\omega t}\delta t$\\
	
	$\oint_L$\\
	
	$\oiint_A$\\
	
	$\oiiint_y$\\

	$h_\theta(x) = \theta_0 + \theta_1x_1 + \theta_2x_2 + \cdots +  \theta_n x_n $

\begin{align} 
	(a^2 + b^2) l= c^2\\
	c l= \sqrt[]{(a^2 * b^2)}\\
	A \oplus \normalsize B l= A\overline{B} + \overline{A}B\\
	El valor de: $R_1$ es: $300[\Omega]\\
	Z(X) l= \frac{X- \mu}{\sigma}\\
	\sum_{i = 0}^{n} n l= \frac{n(n+1)}{2}\\
	\sin^2 x + \cos^2x l= 1\\
	F(\omega) l= \int_{\infty}^{\infty} f(t)e^{-j\omega t}\delta t\\
	\oint_L\\
	\oiint_A\\
	\oiiint_y\\
	\h_\theta(x) = \theta_0 + \theta_1x_1 + \theta_2x_2 + \cdots +  \theta_n x_n\\
\end{align}
	

\end{document}