\documentclass[12pt]{article}
\usepackage[utf8]{inputenc}
\usepackage[spanish]{babel}
\usepackage{amsmath} % Paquete para uso matemático
\usepackage{amssymb} % Paquete para uso matemático
\usepackage{graphicx}
\usepackage{caption}
\usepackage{wrapfig}
\usepackage{lipsum}
\usepackage{vmargin}
\usepackage{subcaption}
\usepackage{multirow} % Para crear multiples renglones
\usepackage{booktabs} % vertical rules 
\usepackage{hhline}


	\newcommand{\formulaGeneral}{ $x_1, x_2 = \frac{-b \pm\sqrt{b^2 - 4ac}}{2a}$}
	
\begin{document}
	
	\section{Imágenes}
	Usando el paquete \textit{lipsum} incluir texto junto con las imágenes, mostrar una imagen normal, una imágen con \textit{wrapfigure} y una con \textit{subfigure}. 
	
		\begin{figure}[h] 
			\centering
			\includegraphics[scale = 0.9]{/Users/gabrielramirezgarcia/Desktop/Clases_LaTex/img/Lipsum1}
			\caption{Lipsum cartoon, graffiti shop and gallery}
		\end{figure}
		\lipsum
	
		\newpage
		\begin{wrapfigure}[10] {l} [-6 mm]{0.5\textwidth} 
			\centering
			\includegraphics[scale = 0.3]{/Users/gabrielramirezgarcia/Desktop/Clases_LaTex/img/Lipsum3}
			\caption{Lipsum cartoon multicolor}
		\end{wrapfigure}
		\lipsum
		
		\newpage
		\lipsum[1-3]
		\begin{figure}[ht]
			\centering
			\begin{subfigure} [b]{0.4\textwidth}
				\includegraphics[scale = 0.17]{/Users/gabrielramirezgarcia/Desktop/Clases_LaTex/img/Lipsum2}
				\caption{Lipsum logo num1}
			\end{subfigure}
			\hspace{15mm}
			\begin{subfigure} [b]{0.26\textwidth}
					\includegraphics[scale = 0.28]{/Users/gabrielramirezgarcia/Desktop/Clases_LaTex/img/Lipsum4}
				\caption{Lipsum logo num2}
			\end{subfigure}
			\caption{Logos of LIPSUM}
		\end{figure}
		\lipsum[4-6]
		
		
	\section{Tablas}
	Tabla que yo hice 
		\begin{table}[th]
			\centering
			\begin{tabular}{|c|c|}  \hline
				7C0                  &  hexadecimal  \\
				3700                &  octal  \\ \hhline{~-} 
				% \begin{tabular}{c} binary \\ \hline binario
				% cline {2-2} tambien puede ser usada esta opcion en lugar de 
				\multirow{2}{*}{111111000000}  &  binary \\  \cmidrule(lr){2-2}  &  binario \\ \hline\hline
				1984                 &  decimal  \\	\hline
			\end{tabular}
		\end{table}
			% addlinespace %\cmidrule{2-2} %\hhline{~-~}
			
			Tabla que hicimos en la clase 4 para resolver las dudas 
		\begin{table}[th]
				\begin{tabular}{|c|c|}  \hline
				7C0                  &  hexadecimal  \\
				3700                &  octal  \\ \hhline{~-} 
				111111000000 & 	\begin{tabular}
										{c} binary \\ \hline binario
											\end{tabular} 
										\\ \hline \hline
				1984                 &  decimal  \\	\hline
			\end{tabular}
		\end{table}
			
		\begin{table}[th]
			\centering
			\begin{tabular}{|c||c|c|}  \hline
				1&uno&one\\	\hline
				2&  &two\\
				3&tres&three\\ \hline \hline
			\end{tabular}
		\end{table}
	
		\begin{table}[th]
			\centering
			\begin{tabular}{|c|c|c|c|c|} \hline
				\multicolumn{5}{|c|}{\textbf{Título}}\\\hline
				\multicolumn{4}{|c|}{Entradas}&\multicolumn{1}{|c|}{Salidas}\\\hline
				A&B&C&D&S\\\hline
				0&0&0&1&1\\\hline
				0&1&1&0&0\\\hline
			\end{tabular}
		\end{table}
	
	\newpage
	\section{Creacion de comando}
	\textbf{Ecuacion de la formula general}\\	
 
 En lecciones anteriores se ha mostrado algunas técnicas para resolver ecuaciones de segundo grado, las cuales van desde el tanteo hasta la factorización. Sien embargo, existen ecuaciones cuadráticas que no pueden resolverse con dichas técnicas. Existe una técnica llamada \textbf{formula general para resolver ecuaciones cuadráticas de segundo grado} que funcionan con cualquier ecuación.
 
 Puedes resolver una ecuación cuadrática \textbf{completando el cuadrado} reescribiendo parte de la ecuación como un trinomio cuadrado perfecto. Si completas el cuadrado de una ecuación genérica $ax^2 + bx + c = 0$ y luego resuelves $x$, encuentras que \formulaGeneral, esta ecuación se conoce como ecuación cuadrática. Esta formula es muy útil para resolver ecuaciones cuadráticas que son difíciles o imposibles de factorizar y usarlas puede ser mas rápido que completar el cuadrado. La formula cuadrática puede usarse para resolver cualquier ecuación de la forma $ax^2 + bx + c = 0$.  Recuerda que una raíz cuadrada posee siempre dos valores, uno positivo y uno negativo. De manera que cuando utilices la formula general debes completar ambos signos por separado.\\
 	
 		%\begin{gather}
 		%	$\formulaGeneral$
 		%\end{gather}
 	
\end{document}