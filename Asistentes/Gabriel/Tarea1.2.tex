\documentclass[12pt, a3paper]{article}
\usepackage[utf8]{inputenc}
\usepackage{xcolor} 
\usepackage{soul} 
\usepackage[T1]{fontenc}
\usepackage{anysize}
\usepackage{ulem}
\usepackage[spanish]{babel}

\title{\Huge\textbf{¿Qué puedo hacer con \LaTeX?}}
\author{\Large\textbf{Gabriel Ramírez García}}
\date{\textit{\today}}

\marginsize{3cm}{3cm}{1cm}{1cm}

\begin{document}
\maketitle

El aprendizaje de los comandos básicos y avanzados del \textbf{\LaTeX} para la escritura de textos lo podría utilizar para realizar la escritura de los manuscritos científicos de investigaciones originales que se realizan en el laboratorio al cual pertenezco. Actualmente las revistas científicas aceptan el envió de los manuscritos en formato \textbf{\LaTeX}. Anteriormente solo era posible enviar archivos en formato \textbf{PDF} o \textbf{WORD}.  Además, dado que realizo divulgación científica, la escritura de dichos artículos cortos podría también realizarlos utilizando \textbf{\LaTeX}. Finalmente, dado que estoy en proceso de escritura de mi tesis doctoral, tengo toda la intención de escribirla en esta plataforma \textbf{\LaTeX}, ya que la mayoría de los investigadores que trabajan en el campo del conocimiento en el cual estoy trabajando de una y otra forma, prefieren la escritura de los textos científicos en formato \textbf{\LaTeX}, ya que la escritura de expresiones matemáticas es mas fácil y estable al realizarla utilizando este sistema de composición de textos.
Considero que esos serian los usos más próximos y principales que podría realizar una vez finalizado el curso.\\


Dado que soy neófito en el uso de \textbf{\LaTeX} el temario parece ser completo y adecuado, sin embargo, mi principal interés es en la incorporación de expresiones algebraicas así como la de figuras y tablas dentro del texto. Las figuras en los textos científicos en biomedicina son esenciales al momento de la redacción de los resultados. Por lo que poder crear textos con imágenes es importante para mi. Otra cosa que es de mi interés, es poder aprender la forma de citar artículos científicos dentro del texto en si. Así como saber si \textbf{\LaTeX} puede estar vinculado con otras plataformas para citar artículos científicos como \textbf{EndNote} entro otros. 

\end{document}