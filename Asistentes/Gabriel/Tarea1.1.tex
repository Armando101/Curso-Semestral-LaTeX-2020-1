\documentclass[12pt, a3paper]{article}
\usepackage[utf8]{inputenc}
\usepackage{xcolor} %This packages provides a common set of commands for colour manipulation in the whole text like color font. 
\usepackage{soul} %This packages provides a common set of commands for highlighting the text even the {ulem} package also do this. 
\usepackage[T1]{fontenc} %This package specify the font encodings. A font encoding is a mapping of the character codes to the font glyphs that are used to typeset your output.
\usepackage{anysize}
\usepackage{ulem} %Package to use \uline command because \underline command doesn't make linebreaks the text and it encloses its argument in a horizontal box.

\title{\Huge\textbf{Star Trek}}
\author{gabboramirez7}
\date{September 2019}

\marginsize{2.5cm}{2.5cm}{2cm}{2cm}

\begin{document}
    \begin{center}
        {\fontfamily{lmss}\selectfont
        \Huge\textbf{Neuroimaging}
        }\\
        \vspace{0.3cm}
        {\fontfamily{qtm}\selectfont
        \huge\textbf{\textcolor{blue}{Magnetic Resonance Imaging}}
        }\\
    \end{center}
    
    \vspace{0.5cm}
   Magnetic resonance imaging (MRI) is a medical imaging technique used in radiology to form pictures of the anatomy and the physiological processes of the body. MRI scanners use strong magnetic fields, magnetic field gradients, and radio waves to generate images of the organs in the body. MRI does not involve X-rays or the use of ionizing radiation, which distinguishes it from CT or CAT scans and PET scans. \uline{Magnetic resonance imaging is a medical application of nuclear magnetic resonance (NMR). NMR can also be used for imaging in other NMR applications.}
    
    \vspace{0.5cm}
    \begin{flushleft}
         \underline{\Large\textbf{How does MRI work?}}
    \end{flushleft}
    
     MRIs employ powerful magnets which produce a \textbf{\textcolor{red}{strong magnetic field that forces protons in the body to align with that field. When a radiofrequency current is then pulsed through the patient}}, the protons are stimulated, and spin out of equilibrium, straining against the pull of the magnetic field. When the \textbf{radiofrequency} field is turned off, the MRI sensors are able to detect the energy released as the protons realign with the magnetic field. The time it takes for the protons to realign with the magnetic field, as well as the amount of energy released, changes depending on the environment and the chemical nature of the molecules. Physicians are able to tell the difference between various types of tissues based on these magnetic properties. To obtain an MRI image, a patient is placed inside a large magnet and must remain very still during the imaging process in order not to blur the image. \hl{Contrast agents (often containing the element Gadolinium) may be given to a patient intravenously before or during the MRI to increase the speed at which protons realign with the magnetic field}. The faster the protons realign, \textsc{the brighter the image}.
     
     \vspace{0.5cm}
     \begin{flushleft}
     \underline{\textit{What is MRI used for?}\textsc{Application and common use}}\\
     \end{flushleft}
      
     MRI scanners are particularly well suited to image the non-bony parts or soft tissues of the body. They differ from computed tomography (CT), in that they do not use the damaging ionizing radiation of x-rays. The brain, spinal cord and nerves, as well as muscles, ligaments, and tendons are seen much more clearly with MRI than with regular x-rays and CT; for this reason MRI is often used to image knee and shoulder injuries. In the brain, MRI can differentiate between white matter and grey matter and can also be used to diagnose aneurysms and tumors. Because MRI does not use x-rays or other radiation, it is the imaging modality of choice when frequent imaging is required for diagnosis or therapy, especially in the brain. However, MRI is more expensive than x-ray imaging or CT scanning. One kind of specialized MRI is functional Magnetic Resonance Imaging (fMRI.) This is used to observe brain structures and determine which areas of the brain “activate” (consume more oxygen) during various cognitive tasks. It is used to advance the understanding of brain organization and offers a potential new standard for assessing neurological status and neurosurgical risk.
     
\end{document}
