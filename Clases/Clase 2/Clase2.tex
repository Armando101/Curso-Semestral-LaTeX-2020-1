\documentclass{article}

\usepackage[spanish]{babel}
\usepackage[utf8]{inputenc}
\usepackage{xcolor}
\usepackage{ulem}
\usepackage{soul}
\usepackage{graphicx}
\usepackage{vmargin}

% Paquetes a utilizar en ambiente matemático
\usepackage{amsmath, amssymb}

\definecolor{MiColor}{rgb}{1,0.5, 0}

\begin{document}
	
	\renewcommand{\figurename}{Imagen}
	\renewcommand{\listfigurename}{Figuras}
	
	\listoffigures
	\newpage
	\textcolor{MiColor}{Hola mundo}\\
	\uline{Hola mundo}\\
	\uwave{Hola Mundo}\\
	\xout{Hola Mundo}\\
	\st{Hola Mundo}
	
	\section{Ambiente Matemático}
	
	
	En \LaTeX  tenemos tres formas básicas de incluir ecuaciones en nuestros textos, la primera es \textit{inline} Si $ a> b $ y $ b>c \therefore a>c \ldots$\\
	
	$ a_n=a_{n+1} + a_{n+2} + a_{n+3}^2 $\\
	$ f(x)=e^{(\alpha+x)}-c $\\
	$ P(x)=A + \Lambda $
	
	$ \int_{-\infty}^{\infty} $

\textbf{Modo display}

\[
	\int_{-\infty}^{\infty}P(X) dx
\]

\[
	P(x)=\frac{Q(x)}{R(X)}
\]

\textbf{Ambiente equation}

\begin{equation}	
	\sqrt{x} \Leftrightarrow x^{\frac{1}{2}}
	\label{eq: Raiz}		
\end{equation}

Según la ecuación \ref{eq: Raiz} la raiz cuadrada de x es x a la un medio

\begin{align}
	E(s)&=R(s)-B(s)\\
	Y(s)&=E(s)G(s)\\
	G(s)&=\frac{Y(s)}{E(s)}\label{eq:Ecuacion2}
\end{align}

Segun la ecuación \ref{eq:Ecuacion2}

\section{Signos de agrupación}

\{a+b\}\\
$ \langle a+b \rangle \\
\lvert a+b \rvert\\
\lVert a+b \rVert\\
\lfloor a+b \rfloor\\
\lceil a+b \rceil$

\[
	h_\theta(x)=g\Bigg(\frac{1}{1+e^{(-\theta^Tx)}}\Bigg)
\]

\section{Matrices}

\[
	W(f_1, f_2)(x)=
	\begin{pmatrix}
	x^2 & x|x|\\
	2x & \frac{2x^2}{|x|}
	\end{pmatrix}
\]

\[
	\sqrt[3]{5}
\]

\section{Incluir imágenes}

\begin{figure}[h!]
	\centering
	\includegraphics[scale=0.5, angle=0]{img/unam}
	\caption{Escudo UNAM}
	\label{fig:UNAM}
\end{figure}

El escudo de la UNAM se muestra en la figura \ref{fig:UNAM}

\end{document}