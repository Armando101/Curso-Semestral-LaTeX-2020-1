\documentclass{article}

\usepackage[utf8]{inputenc}
\usepackage[spanish]{babel}
\usepackage{amsmath}
\usepackage{amssymb}
\usepackage{graphicx}
\usepackage{caption}
\usepackage{wrapfig}

\usepackage{lipsum}
\usepackage{vmargin}
\usepackage{subcaption}
\usepackage{colortbl}

\spanishdecimal{.}

\newcommand{\Vector}{x_1, x_2, \dots, x_n}
\newcommand{\Vectorr}[1]{#1_1, #1_2, \dots, #1_n}
\newcommand{\Vectorrr}[2]{#1_1, #1_2, \dots, #1_#2}


\newcommand{\nuevoVector}[2][x]{#1_1, #1_2, \dots, #1_#2}




\begin{document}
	
	\renewcommand{\figurename}{Imagen}
	\renewcommand{\tablename}{Tabla}
	
	\renewcommand{\listtablename}{Índice de tablas}
	\renewcommand{\listfigurename}{Índice de imágenes}
	\renewcommand{\contentsname}{Índice general}
	
%	\renewcommand{\tabcolsep}{10mm}
%	\renewcommand{\arraystretch}{5}
%	\renewcommand{\arrayrulewidth}{1mm}
	
	\tableofcontents
	\listoffigures
	\listoftables
	
	\section{Creación de comandos}
	Aquí va un Vector $ \Vector $\\
	Aquí hay otro vector $ \Vectorr{y} $\\
	 aquí está otro vector $ \Vectorr{W} $\\
	aquí está otro vector $ \Vectorrr{X}{W} $\\
	aquí está otro vector $ \nuevoVector{t} $\\
	\\
	
	$ 5.38 $
	
	\begin{gather}
		 \sqrt[5]{x}\\
		\prod_{i=0}^{n}\\\label{eq:Raiz}
		\sqrt{x}\Leftrightarrow
	\end{gather}
	
	Cómo se puede ver en la ecuación \ref{eq:Raiz}
	
	\section{Imágenes}
	
	\begin{figure}[h]
		\centering
		\includegraphics[scale=0.5]{img/UNAM}
		\caption{Logo UNAM}
		\label{Fig:Logo}
	\end{figure}
	

En la figura \ref{Fig:Logo} se muestra el escudo de la UNAM
	\newpage
	\begin{wrapfigure}[13]{l}[0mm]{0.35\textwidth}
		\centering
		\includegraphics[scale=0.5]{img/python}
		\caption{Python}
	\end{wrapfigure}
	\lipsum
	
	\begin{figure}[ht]
		\centering
		\begin{subfigure}[t]{0.35\textwidth}
			\includegraphics[scale=0.5]{img/latex}
			\caption{Logo de \LaTeX}
			\label{Fig:LaTeX}
		\end{subfigure}
	\hspace{10mm}
		\begin{subfigure}[t]{0.35\textwidth}
			\includegraphics[scale=0.25]{img/csh}
			\caption{Logo de \LaTeX}
			\label{fig:LogoCsh}
		\end{subfigure}	
	\caption{Logos}
\end{figure}
	El logo de \LaTeX se encuentra en la figura \ref{Fig:LaTeX}
\section{Ambiente tabular}

\begin{table}[h]
	\centering
	\begin{tabular}{lr|c}
		X	&	Y	&	AND\\
		\hline
		0	& 	0	&	0\\
		0	& 	1	&	0\\
		1	& 	0	&	0\\
		1	& 	1	&	1\\
	\end{tabular}
\caption{Compuerta lógica AND}
\label{Tab:AND}
\end{table}

En la tabla \ref{Tab:AND} se muestra la tabla de verdad de la compuerta lógica AND

\begin{wraptable}{l}{0.4\textwidth}
	\centering
	\begin{tabular}{lr|c}
		X	&	Y	&	AND\\
		\hline
		0	& 	0	&	0\\
		0	& 	1	&	0\\
		1	& 	0	&	0\\
		1	& 	1	&	1\\
	\end{tabular}
	\caption{Wrap table}
\end{wraptable}
\lipsum

\begin{table}[h]
	\centering
	\begin{tabular}{|c|c|}
		\hline
		\multicolumn{2}{|c|}{\textbf{Primera Medición}}
		\\\hline
		Nivel(cm)&Ángulo$\theta$\\\hline
		0&28\\\hline
		1&30\\\hline
		2&33\\\hline
		3&35\\\hline
		4&37\\\hline
		5&38.5\\\hline
		6&41\\\hline
		7&44\\\hline
	\end{tabular}
	\hspace{1cm}
	\begin{tabular}{|c|c|}
		\hline
		\multicolumn{2}{|c|}{\textit{Primera Medición}}
		\\\hline
		Nivel(cm)&Ángulo$\theta$\\\hline
		0&28\\\hline
		1&30\\\hline
		2&33\\\hline
		3&35\\\hline
		4&37\\\hline
		5&38.5\\\hline
		6&41\\\hline
		7&44\\\hline
	\end{tabular}
\hspace{1cm}
\begin{tabular}{|c|c|}
	\hline
	\multicolumn{2}{|c|}{Primera Medición}
	\\\hline
	Nivel(cm)&Ángulo$\theta$\\\hline
	0&28\\\hline
	1&30\\\hline
	2&33\\\hline
	3&35\\\hline
	4&37\\\hline
	5&38.5\\\hline
	6&41\\\hline
	7&44\\\hline
\end{tabular}
	\caption{Mediciones}
\end{table}

\begin{table}[h]
	\centering
	\begin{tabular}{|c|c|c|c|c|c|}
		\hline
		\multicolumn{6}{|c|}{Extremo a Extremo}\\\hline
		\multicolumn{3}{|c|}{Primera Medición}&
		\multicolumn{3}{c|}{Segunda Medición}\\\hline
		Deflexión  $\theta[^\circ]$&Resistencia $[\Omega]$& Fuerza [N]&Deflexión  $\theta[^\circ]$&Resistencia $[k\Omega]$& Fuerza [N]\\\hline
		0	&227	&0		&0		&272	&0\\\hline
		3	&225.2	&0.25	&13		&272.2	&0.25\\\hline
		6	&225	&0.5	&10		&275	&0.5\\\hline
		9	&225	&0.75	&12		&273	&0.75\\\hline
		11	&226.3	&1.1	&11		&272	&1\\\hline
		12	&226.5	&1.25	&15		&273.3	&1.25\\\hline
		13.5&226.5	&1.5	&18		&273.4	&1.5\\\hline
		15	&227.2	&1.75	&20		&274.5	&1.75\\\hline
		16	&228	&2		&24		&275.9	&2\\\hline
		19	&228.6	&2.25	&26		&282	&2.25\\\hline
	\end{tabular}
	\caption{Datos tomados al medir la resistencia de Extremo a Extremo}
	\label{Tab:E-E}
\end{table}

\lipsum
\lipsum

\end{document}