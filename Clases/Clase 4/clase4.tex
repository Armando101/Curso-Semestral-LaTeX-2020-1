\documentclass{article}

\usepackage[hidelinks]{hyperref}
\usepackage{color}
\usepackage{xcolor}
\usepackage{lipsum}
\usepackage{vmargin}
\usepackage{nameref}
\usepackage[most]{tcolorbox}
\usepackage{multicol}

\definecolor{myBlue}{HTML}{1C7FBB}
\definecolor{applegreen}{rgb}{0.55, 0.71, 0.0}
\definecolor{myGreen}{HTML}{36A736}

\begin{document}
	
	\section{Una sección} \label{UnaSeccioon}
	\lipsum
	\section{Otra sección}
	\lipsum
	
	\section{Referencias}
	En la página de la \textit{UNAM} puedes encontrar el calendario académico \\\url{https://unam.mx/}.\\ En la página de
	\textcolor{myBlue}{\href{https://proteco.mx}{PROTECO}}
	puedes encontrar una gran variedad de cursos de tecnología en computación.
	
	En la sección de \textcolor{myBlue}{\nameref{Resultados}} hay texto aleatorio
	
	\section{Resultados}\label{Resultados}
	\lipsum
	
	\section{Fuentes}
	\ttfamily Este es un teto en \LaTeX con la fuente typewriter\\
	\sffamily Este es un texto en \LaTeX con la fuente sans serif\\
	\rmfamily Este es un testo en \LaTeX con la fuente roman
	
	\section{Cajas}
	\subsection{mbox y makebox}
	\mbox{Este es un texto dentro de un mbox}\\
	\makebox[0.8\textwidth][r]{Este es un texto dentro de un makebox}
	
	\subsection{fbox y framebox}
	\fbox{Este es un texto dentro de un fbox}\\
	\framebox[0.8\textwidth][c]{Aquí hay un texto dentro de un framebox, Este es un texto muy largo dentro de un framebox }
	
	\subsection{parbox}
	\fbox{\parbox{0.5\textwidth}{Aquí hay un texto dentro de un parbox, este texto es muy largo y debe de caber dentro de mi caja}}
	
	\subsection{Cajas con coles}
	
	\colorbox{myBlue}{Este es un texto dentro de una caja azul}\\
	\fcolorbox{red}{myGreen}{Este es un texto dentro de una caja verde}\\
	\fcolorbox{applegreen}{myGreen}{\parbox{\textwidth
		}{Este es un texto dentro de una caja verde con borde verde. Puedo incluir ecuaciones matemáticas $$ \int{dx}=x+C $$}}
	
	\begin{tcolorbox}[
		colback = orange!30, %color de fondo
		colframe = gray, 
		arc = 3mm, 
		width = 0.7\textwidth,
		height = 40pt,	
		boxrule = 5pt,
		enhanced jigsaw,
		drop shadow = {green}
		]
		Texto en un colorbox simple
	\end{tcolorbox}
	
	\begin{tcolorbox}[
		colback =blue!25, 
		title = aquí hay una caja,
		arc=3mm,
		colframe=blue!85!black,
		fonttitle=\ttfamily,
		coltext=black,
		coltitle=white,
		drop shadow = {black!30},
		enhanced jigsaw,
		boxrule=3pt
		]
		
		\lipsum[1-2]
		La masa de la región S está definida por:
		\[
			\int\int_{S}\rho(x,y)dxdy
		\]
		Donde $ \rho(x,y) $ es la densidad de la superficie
	\end{tcolorbox}
	
	\section{Columnas}
	
	Colocando dos columnas de texto
	\twocolumn
	\lipsum
	\onecolumn
	\lipsum
	
	
	\begin{multicols}{3}
		\lipsum
		\lipsum
	\end{multicols}
\newpage
	\section{Reglas}
	
	Un ejemplo de una regla\\
	Regla: \rule{\textwidth}{2pt}\\
	Otra regla: \rule{100pt}{5pt}\\
	Una regla más: \rule[-5mm]{100pt}{7pt}\\
	\noindent \rule{100pt}{5pt}\\
	Una regla vertical \rule{7pt}{100pt}\\
	Un cuadrado \rule{5pt}{5pt}
	
	
\end{document}