\documentclass{beamer}
\usepackage[spanish]{babel}
\usepackage[utf8]{inputenc}

\usetheme{Luebeck}
\usecolortheme{orchid}

\author{Armando Rivera}
\date{\today}
\title{Mi primera presentación con \LaTeX}
\institute[UNAM]{Universidad Nacional Autónoma de México\\Facultad de Ingeniería}

\begin{document}
	\begin{frame}
		\maketitle
	\end{frame}
	\begin{frame}{Índice}
		\tableofcontents
	\end{frame}

\section{Introducción}
	\begin{frame}{Introducción}{Sub Título}
	Aquí hay texto que puedo colocar en una diapositiva
	\end{frame}
	\begin{frame}{Introducción}{Segundo frame}
	Aquí hay texto que puedo colocar en un segundo frame
	\end{frame}
\section{bloques}
	\begin{frame}{Bloques}{block}
		Aquí tenemos un bloque simple
		\begin{block}{Pitágoras}
			$  a^2 + b^2 = c^2 $
		\end{block}
	\end{frame}
\begin{frame}{Bloques}{Alertas}
Aquí hay una alerta
\begin{alertblock}{CUIDADO}
	$ (a+b)^2 \neq a^2 + b^2 $
\end{alertblock}
\end{frame}
\begin{frame}{Bloques}{Ejemplos}
	\begin{exampleblock}{Ejemplo}
		Aquí tenemos un ejemplo
	\end{exampleblock}
\end{frame}
\section{Diapositivas dinámicas}
\begin{frame}{Diapositivas dinámicas}
\transblindshorizontal
Este frame es un ejemplo de transición\pause
También podemos realizar pausas
\end{frame}

{
	\usebackgroundtemplate{
		\includegraphics[width=\paperwidth, height=\paperheight]{img/fondo}}
	
		\begin{frame}{Fondo de diapositivas}
			Esta diapositiva tiene el fondo cambiado
		\end{frame}	
}

\end{document}