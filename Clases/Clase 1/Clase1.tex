% Preambulo 

% Tipo de documento
%\documentclass[12pt,a4paper,twoside]{article}
\documentclass{article}
%Paquetes a utilizar
\usepackage[utf8]{inputenc}
\usepackage[spanish]{babel}
\usepackage{lipsum}
\usepackage{graphicx}
\usepackage{anysize}

% Metadatos
\title{Introduccción al curso de \LaTeX}
\author{Armando Rivera}
\date{\today}

\marginsize{2cm}{2cm}{2cm}{2cm}

\begin{document}

\maketitle
\newpage
\tableofcontents
\newpage

%\hline
\rule{\textwidth}{0.5pt}
\begin{abstract}
    Este documento ilustra los principales comandos de latex
\end{abstract}


\section{Objetivo}

El objetivo  de esta practica es involucrar a las personas en latex\\
El objetivo de esta practica es involucrar a las personas en latex

\section*{Introduccion}

Latex puede servir para crear\{ documentos \textit{cientificos} y \textbf{técnicos} solo tienes que $c^2$ \$ \textbf{\underline{intentarlo}}, \textsc{animate } 


\addcontentsline{toc}{section}{Desarrollo}
\section*{Desarrollo}

\large Hola mi nombre es Rodrigo\\[2cm]
Esta es la primera clase de Latex \\
\normalsize El tamaño de la letra fue restaurado\\
\Large \Huge  Este es un texto muy grande \normalsize

\section{Resultados}

\begin{center}
    Este es un texto que esta centrado con el ambiente center
\end{center}

\begin{flushleft}
    \lipsum
\end{flushleft}

\begin{flushright}
    Este es un texto alineado a la derecha
\end{flushright}

%\vspace{2cm}

\section{Conclusiones}

\subsection{Conclusiones invidivuales}

\begin{itemize}
    \item Hoy aprendimos a crear listas sin orden
    \item Apredimos acerca del formato de texto
    \begin{itemize}
        \item tamaños de letra
        \begin{itemize}
            \item Huge
        \end{itemize}
    \end{itemize}
    
    
\end{itemize}
\newpage
\subsection{Conclusiones en equipo}
    \renewcommand{\labelenumii}{\Roman{enumii}}
    \renewcommand{\labelenumiii}{\Alph{enumiii}}
    \begin{enumerate}
        \item Todos los integrantes del curso son de la UNAM
        \item Preferimos Latex que word
        \item Los compiladores que se pueden hacer para Latex son
        \begin{enumerate}
            \item Miktex
            \item Mactex
            \item TexLive
        \end{enumerate}
        \item Los integrantes del curso son
        \begin{enumerate}
            \item Jorge
            \begin{enumerate}
                \item Procendia: UNAM
                \item Procendia: UNAM
            \end{enumerate}
            \item Cassandra
            \item Ariel
            \item Gabriel
        \end{enumerate}
        \item Este texto es demasiado \\grande y por lo tanto no cabe en una linea de la viñeta
        \item $\frac{12}{6} $
    \end{enumerate}
    
    \renewcommand{\labelenumi}{\Roman{enumi}}
    \begin{enumerate}
        \item Frutas
        \item harina
        \item Marcadores
    \end{enumerate}


\end{document}

El texto después de \end{document no se toma en cuenta} y \LaTeX lo ignora